\batchmode
\makeatletter
\def\input@path{{/accounts/gen/vis/paciorek/research/jmac/composition_paper//}}
\makeatother
\documentclass[12pt]{article}\usepackage[]{graphicx}\usepackage[]{color}
%% maxwidth is the original width if it is less than linewidth
%% otherwise use linewidth (to make sure the graphics do not exceed the margin)
\makeatletter
\def\maxwidth{ %
  \ifdim\Gin@nat@width>\linewidth
    \linewidth
  \else
    \Gin@nat@width
  \fi
}
\makeatother

\definecolor{fgcolor}{rgb}{0.345, 0.345, 0.345}
\newcommand{\hlnum}[1]{\textcolor[rgb]{0.686,0.059,0.569}{#1}}%
\newcommand{\hlstr}[1]{\textcolor[rgb]{0.192,0.494,0.8}{#1}}%
\newcommand{\hlcom}[1]{\textcolor[rgb]{0.678,0.584,0.686}{\textit{#1}}}%
\newcommand{\hlopt}[1]{\textcolor[rgb]{0,0,0}{#1}}%
\newcommand{\hlstd}[1]{\textcolor[rgb]{0.345,0.345,0.345}{#1}}%
\newcommand{\hlkwa}[1]{\textcolor[rgb]{0.161,0.373,0.58}{\textbf{#1}}}%
\newcommand{\hlkwb}[1]{\textcolor[rgb]{0.69,0.353,0.396}{#1}}%
\newcommand{\hlkwc}[1]{\textcolor[rgb]{0.333,0.667,0.333}{#1}}%
\newcommand{\hlkwd}[1]{\textcolor[rgb]{0.737,0.353,0.396}{\textbf{#1}}}%

\usepackage{framed}
\makeatletter
\newenvironment{kframe}{%
 \def\at@end@of@kframe{}%
 \ifinner\ifhmode%
  \def\at@end@of@kframe{\end{minipage}}%
  \begin{minipage}{\columnwidth}%
 \fi\fi%
 \def\FrameCommand##1{\hskip\@totalleftmargin \hskip-\fboxsep
 \colorbox{shadecolor}{##1}\hskip-\fboxsep
     % There is no \\@totalrightmargin, so:
     \hskip-\linewidth \hskip-\@totalleftmargin \hskip\columnwidth}%
 \MakeFramed {\advance\hsize-\width
   \@totalleftmargin\z@ \linewidth\hsize
   \@setminipage}}%
 {\par\unskip\endMakeFramed%
 \at@end@of@kframe}
\makeatother

\definecolor{shadecolor}{rgb}{.97, .97, .97}
\definecolor{messagecolor}{rgb}{0, 0, 0}
\definecolor{warningcolor}{rgb}{1, 0, 1}
\definecolor{errorcolor}{rgb}{1, 0, 0}
\newenvironment{knitrout}{}{} % an empty environment to be redefined in TeX

\usepackage{alltt}
\usepackage[T1]{fontenc}
\usepackage[latin9]{inputenc}
\usepackage{geometry}
\geometry{verbose,tmargin=1in,bmargin=1in,lmargin=1in,rmargin=1in}
\usepackage{array}
\usepackage{amsmath}
\usepackage{graphicx}
\usepackage[authoryear]{natbib}

\makeatletter

%%%%%%%%%%%%%%%%%%%%%%%%%%%%%% LyX specific LaTeX commands.
%% Because html converters don't know tabularnewline
\providecommand{\tabularnewline}{\\}

%%%%%%%%%%%%%%%%%%%%%%%%%%%%%% User specified LaTeX commands.
\usepackage{times}
\usepackage{graphics,natbib}

\newcommand{\matern}{Mat\'{e}rn }
\newcommand{\N}{\mathcal{N}}

\makeatother
\IfFileExists{upquote.sty}{\usepackage{upquote}}{}
\begin{document}


\title{Statistically-estimated tree composition for the northeastern United
States \\
at the time of European settlement}
\maketitle
\begin{quote}
Christopher J. Paciorek 1{*}, Simon J. Goring 2\&, Andrew Thurman
3\&, Charles V. Cogbill 4, \\
John W. Williams 3,5, David J. Mladenoff 6, Jody A. Peters 7, Jun
Zhu 8, \\
and Jason S. McLachlan 7 \\
{[}author order to be discussed{]}\\
{[}please tell me if you do/don't want your middle initial{]}

{[}let me know if there are any issues with how I've listed your affiliation{]}

1 Department of Statistics, University of California, Berkeley, California,
USA

2 Department of Geography, University of Wisconsin, Madison, Wisconsin,
USA

3 Department of Statistics, University of Iowa, Iowa City, Iowa, USA

4 Harvard Forest, Harvard University, Petersham, Massachusetts, USA

5 Center for Climatic Research, University of Wisconsin, Madison,
Wisconsin, USA

6 Department of Forestry and Wildlife Ecology, University of Wisconsin,
Madison, Wisconsin, USA

7 Department of Biological Sciences, University of Notre Dame, Notre
Dame, Indiana, USA

8 Department of Statistics, University of Wisconsin, Madison, Wisconsin,
USA

{*} Corresponding author

E-mail paciorek@stat.berkeley.edu (CJP)

\& These authors contributed equally to this work. 
\end{quote}
TODO:

focus on PLOS One, but Nature's Scientific Data is a possibility

check journal requirements for archiving and presentation; talk to
Jody/Ann about archiving/DOI



\begin{abstract}
We present a data product of the estimated composition of tree taxa
at the time of European settlement of the northeastern United States
and the statistical methodology used to produce the product. Composition
is defined as the proportion of stems larger than approximately 20
cm diameter at breast height in 22 taxonomic groupings, generally
at the genus level. The data come from settlement survey records that
provide raw data that are transcribed and then aggregated spatially,
giving count data. The domain is divided into two regions, eastern
(Maine to Ohio) and western (Indian to Minnesota). Public Land Survey
point data in the western region are aggregated to a regular 8 km
grid, while data in the eastern region, from Town Proprietor Surveys,
is aggregated at the township level in irregularly-shaped local administrative
units. The product is based on a Bayesian statistical model fit to
the count data that estimates composition on a regular 8 km grid.
The statistical model allows us to estimate composition at locations
with no data and to smooth over noise caused by limited counts in
locations with data. Critically, it also allows us to quantify uncertainty
in our composition estimates. We expect this data product to be useful
for understanding the state of vegetation in the northeastern United
States prior to large-scale European settlement. In addition to specific
regional questions, the data product can also serve as a baseline
against which to investigate how forests and ecosystems change after
intensive settlement.
\end{abstract}

\section{Introduction}

Historical datasets provide critical context to understand forest
ecology. They allow researchers to define 'baseline' conditions for
conservation management, to understand ecosystem processes at decadal
and centennial scales, and, particularly in regions with widespread
land use change, to understand the extent to which forests after conversion
and regeneration differ from the original forest cover.

Euro-American settlement and subsequent land use change occured in
a time transient fashion across North America, and land surveys that
provide vegetation information followed a similar pattern. Early surveys
(from 1620 until 1825) in the northeastern United States provide areally-aggregated
data at the township level \citep{cogbill2002forests,thompson2013four};
we refer to these as the Town Proprietor Survey (TPS). These surveys
provide composition of major tree species using sometimes inconsistent
common names. Later surveys, after the establishment of the U.S. Public
Land Survey System (PLS) - from 1832 to 1907 - provide point-level
data along a regular grid, with one mile spacing \citet{bourdo1956review,schulte2001original,goring2015composition},
again with sometimes inconsistent common names. Survey instructions
during the PLS varied through time and by point type. This requires
the application of spatially-varying correction factors \citep{cogbill2015corrections,goring2015composition}
to accurately assess stem density, basal area and biomass from the
early settlement records but has limit impact when estimating composition.

Logging, agriculture, and abandonment have left an indelible mark
on forests in the northeastern United States \citet{foster1998land,rhemtulla2009legacies,thompson2013four,goring2015composition}.
However most studies have assessed these effects in individual states
or smaller domains \citep{rhemtulla2009historical,friedman2005regional}
and with varying spatial resolution, from townships (36 square miles)
to forest zones of hundreds or thousands of square miles. \citet{goring2015composition}
provide a dataset for the upper Midwest that is resolved to an 8 x
8 km grid size, providing broad spatial coverage at a spatial scale
that can be compared to the Forest Inventory and Analysis products
\citet{gray2012forest}. Combined with additional, coarsely-sampled
PLS data from Illinois and Indiana and with the TPS data, this gives
us raw data for much of the northeastern United States. However, there
are several limitations of using the raw data that can be alleviated
by the use of a statistical model to develop a statistically-estimated
data product. First, the PLS and TPS data provide only estimates of
within-cell variance that do not account for information from nearby
locations. Second, the available digitized data from Illinois and
Indiana represent a small fraction of those states and missing townships
are common in the TPS data. 

Properly assessing uncertainty in ecological data is imperative to
understanding and modelling ecological processes \citep{cressie2009accounting}.
In this way, a model that can account for the spatial structure of
the underlying PLS and TPS data, and provide reliable estimates of
uncertainty across the northeastern United States, provides a valuable
tool for researchers interested in the ecological structure and function
of forests at longer time scales.


\section{Data}

The raw data were obtained from survey records collated from across
the northeastern U.S. by a number of researchers. For the states of
Minnesota, Wisconsin, Illinois, Indiana, and Michigan (the western
subdomain), data are available at PLS survey point locations and have
been aggregated to a regular 8 km grid in the Albers projection. (Note
that for Indiana and Illinois, at the moment trees are associated
with township centroids and then assigned to 8 km grid cells based
on the centroid but in the near future we will have point locations
available for each tree.) For the states of Ohio, Pennsylvania, New
Jersey, New York and the six New England states (the eastern subdomain),
data are aggregated at the township level. There are also data from
a single township in Quebec and a single township in northern Delaware.
Data are essentially complete in Minnesota, Wisconsin and Michigan,
but data in Illinois and Indiana represent a sample of the full set
of grid cells, with survey record transcription ongoing. Data for
the remaining states are available for a subset of the full set of
townships covering the domain. Fig. \ref{fig:domain} {[}Fig. 1 -
Lyx is having a numbering problem{]} shows the domain, indicating
the grid cells and townships with data.

\begin{figure}
\label{fig:domain}\includegraphics{0_accounts_gen_vis_paciorek_research_jmac_composition_paper_fig1.pdf}

\caption{Spatial domain, with locations with data shown in gray. Locations
are grid cells in western portion and townships in eastern portion.
In addition to locations without data being indicated in white, grid
cells completely covered in water are white (e.g., a few locations
in Minnesota and Wisconsin).}
\end{figure}


Note that surveys occurred over a period of more than 200 years as
European colonists (before U.S. independence) and the United States
settled what is now the northeastern United States. Our estimates
are for the period of settlement represented by the survey data and
therefore are time-transgressive; they do not represent any single
point in time across the domain, but rather the state of the landscape
at the time just before settlement occurred \citep{Whit:1996,Cogb:etal:2002}.

Extensive details on the data are available in \cite{Gori:etal:2015}
and the raw data are available at XXX {[}ask Simon/Charlie about public
access to Paleon composition data{]}. The aggregation into taxonomic
groups is primarily at the genus level, but is at the species level
in some cases of monospecific genera. We model the following 22 taxa
plus an ``other hardwood'' category: Atlantic white cedar (\emph{Chamaecyparis
thyoides}), Ash (\emph{Fraxinus spp.}), Basswood (\emph{Tilia americana}),
Beech (\emph{Fagus grandifolia)}, Birch (\emph{Betula spp.}), Black
gum/sweet gum (\emph{Nyssa sylvatica} and \emph{Liquidambar styraciflua}),
Cedar/juniper (\emph{Juniperus virginiana} and \emph{Thuja occidentalis}),
Cherry (\emph{Prunus spp.}), Chestnut (\emph{Castanea dentata}), Dogwood
(\emph{Cornus spp.}), Elm (\emph{Ulmus spp.}), Fir (\emph{Abies spp.}),
Hemlock (\emph{Tsuga canadensis}), Hickory (\emph{Carya spp.}), Ironwood
(\emph{Carpinus caroliniana} and \emph{Ostrya virginiana}), Maple
(\emph{Acer spp.}), Oak (\emph{Quercus spp.}), Pine (\emph{Pinus spp.}),
Poplar/tulip poplar (\emph{Populus spp.} and \emph{Liriodendron tulipifera}),
Spruce (\emph{Picea spp.}), Tamarack (\emph{Larix laricina}), Walnut
(\emph{Juglans nigra}). Note that in several cases (e.g., black gum/sweet
gum, ironwood, poplar/tulip poplar, cedar/juniper), because of ambiguity
in the common tree names used by surveyors, a group represents trees
from different families and even orders. For the western subdomain
we do not fit statistical models for Atlantic white cedar and chestnut
as these have 0 and 6 trees present, respectively. The taxa grouped
into the other hardwood category are those for which fewer than roughly
2000 trees were present in the dataset; however we include Atlantic
white cedar explicitly despite it only having 336 trees in the dataset.
Given the nature of the data, there is not a firm cutoff on the sizes
of trees included, but the data roughly represent trees greater than
8 inches (\textasciitilde{}20 cm) diameter at breast height (dbh),
but with some trees as small as 4 inches. {[}Charlie,Simon, please
check this wording{]} In Indiana and Illinois there are also 402 trees
and 29 trees, respectively, that are between 1 and 4 inches dbh, but
these will be excluded in the next version of the product. 

There are approximately 860,000 trees from the western subdomain and
420,000 trees from the eastern subdomain. In the western subdomain,
oak is the most common taxon and pine the second most common, while
in the eastern subdomain oak is the most common and beech the second
most common.

Our domain is a rectangle covering all of the states, in the Albers
projection (NAD 1983 Albers Great Lakes and St. Lawrence), with the
rectangle split into 8 km cells, arranged in a 296 by 180 grid of
cells, with the centroid of the cell in the southeast corner located
at (-71000, 58000). For the modeling of the western subdomain we use
the western-most 146 by 180 grid of cells. For the modeling of the
eastern subdomain we use the eastern-most 180 by 180 grid of cells
and then omit 23 cells in the north and 17 cells in the south outside
the states containing data.


\section{Statistical model}

We fit a Bayesian statistical model to the data, with two primary
goals:
\begin{enumerate}
\item To estimate composition on a regular grid across the entire domain,
filling gaps where no data are available, and
\item To quantify uncertainty in composition at all locations. Even in grid
cells and townships with data, we wish to quantify uncertainty because
the empirical proportions represent estimates of the true proportions
that could be calculated using full population of all the trees in
an areal region.
\end{enumerate}
The result of fitting the Bayesian model via Markov chain Monte Carlo
(MCMC) is a set of representative samples from the posterior distribution
for the composition in the 23 taxonomic groupings at each of the grid
cells. These samples are the data product (described further in the
Data Product section), and can then be used in subsequent analyses.
The mean and standard deviation of the samples for each cell by taxon
pair represent our best estimate (i.e., prediction) of composition
and a Bayesian ``standard error'' quantifying the uncertainty in
the estimate. 


\subsection{Data model}

We start by describing the basic model for those states for which
we have raw data on the 8 km grid and in Section \ref{sub:Model-for-township}
we describe the extension of the model to accommodate data aggregated
at the township level.

The statistical model treats the observations as coming from a multinomial
distribution with a (latent) vector of proportions for each grid cell.
\[
y_{i}\sim\mbox{Multi}(n_{i},\theta(s_{i}))
\]
where $y_{i}$ is the vector of counts for the $P$ taxa at the $i$th
cell, $n_{i}$ is the number of trees counted in the cell, and $\theta(s_{i})$
is the vector of unknown proportions for those taxa at that cell.
Note that we use a standard multinomial distribution without overdispersion
as the set of trees in the dataset is roughly uniformly sampled across
the cells or townships \citep{Gori:etal:2015}.

The proportions, $\theta_{p}(s_{i}),\, p=1,\ldots,P$, are modeled
spatially by a set of $P$ Gaussian spatial processes, one per taxon,
$\alpha_{p}(s_{i}),\, p=1,\ldots,P$. This collection of processes
defines a multivariate spatial process for composition. The $\alpha_{p}(s)$
processes are defined on the 8 km grid, $\alpha_{p}=\{\alpha_{p}(s_{1}),\ldots,\alpha_{p}(s_{m})\}$
for the $m$ grid cells. In Section \ref{sub:Latent-Variable-Model}
we introduce a multinomial probit model that relates the $\alpha_{p}(s)$
processes to the proportion processes, $\theta_{p}(s)$, via the introduction
of latent variables, with an implicit sum-to-one constraint, $\sum_{p=1}^{P}\theta_{p}(s)=1$.



The critical component of the statistical model is the representation
of $\alpha_{p}(s)$ as a spatial process. This process is a prior
structure that serves to smooth across noise in the observations and
allows for interpolation to locations with no data. Apart from the
sum to one constraints, the taxa are considered to be independent
in the prior. We did not want to impose any structure that ties the
different taxa together, as any correlation will likely vary across
space.

In the next section, we consider two spatial models to define the
structure of the $\alpha_{p}(s)$ processes, a standard conditional
autoregressive model \citep{Bane:etal:2003} and a Gaussian Markov
random field (MRF) approximation to a Gaussian process with Matern
covariance \citep{Lind:etal:2011}. 


\subsection{Spatial process models}

MRF models work directly with the precision matrix of the values of
the spatial process, so calculation of the prior density of $\alpha_{p}(s)$
is computationally simple \citep{Rue:Held:2005}, but in situations
where the likelihood is not normal, it can be difficult to set up
effective MCMC algorithms that are able to move in the high-dimensional
space of $\alpha_{p}$. The latent variable representation of Section
\ref{sub:Model-for-township} helps to alleviate this problem.


\subsubsection{Standard conditional autoregressive models}

Our first model is a standard conditional autoregressive (CAR) model
\citep{Bane:etal:2003}. We use a standard form of this model, which
treats the four cardinal neighbors of each grid cell as the neighbors
of the grid cell. The corresponding precision matrix has diagonal
elements, $Q_{ii}$, equal to the number of neighbors for the $i$th
area (i.e., four except for cells on the boundary of the domain),
while $Q_{ik}=-1$ (the negative of a weight of one) when areas $i$
and $k$ are neighbors and $Q_{ik}=0$ when they are not. This gives
the following model for the values of $\alpha_{p}(s_{i})$ collected
as a vector across all of the grid cells, $i=1,\ldots,m$: 
\[
\alpha_{p}\sim\mbox{N}(0,\sigma_{p}^{2}Q^{-})
\]
The use of the generalized inverse notation indicates that $Q$ is
not full-rank, but is of rank $m-1$; this gives an improper prior
on an implicit overall mean for the process values. This specification
is called an \textit{intrinsic conditional autoregression (ICAR)}
and we can write $Q=D-C$ where $C$ is the $m\times m$ adjacency
matrix defining the neighborhood relation of the locations; that is
$(C)_{ik}=1$ if locations $i$ and $k$ are neighbors. The matrix
$D$ is an $m\times m$ diagonal matrix containing the row sums of
matrix $C$ as the diagonal entries $(D)_{ii}={\displaystyle \sum_{k=1}^{m}(C)_{ik}}.$

We refer to this as the \emph{CAR model}.


\subsubsection{Gaussian process approximation}

Gaussian processes (GP) are also standard models for spatial processes \citep{Bane:etal:2003}.
GP models are computationally challenging for large datasets because
of manipulations involving large covariance matrices. Given this,
\cite{Lind:etal:2011} proposed a new framework for using Gaussian
MRFs (GMRFs) as approximations to GPs, based on the use of stochastic
partial differential equations (SPDE).

We consider Gaussian processes in the \matern class, using the following
parameterization of the \matern correlation function as 
\begin{equation}
R(d)=\frac{1}{\Gamma(\nu)2^{\nu-1}}\left(\frac{2\sqrt{\nu}d}{\rho}\right)^{\nu}\mathcal{K}_{\nu}\left(\frac{2\sqrt{\nu}d}{\rho}\right),\label{eq:Matern}
\end{equation}
where $d$ is Euclidean distance, $\rho$ is the spatial range parameter,
and $\mathcal{K}_{\nu}(\cdot)$ is the modified Bessel function of
the second kind, whose order is the smoothness (differentiability)
parameter, $\nu>0$. $\nu=0.5$ gives the exponential covariance.
For any pair of locations, $R(d)$ defines the correlation of the
process, (i.e., $\alpha_{p}(s)$ in our context), as a function of
the distance between the locations. Considering all pairs of locations,
this defines a correlation matrix for all locations of interest. 

The \citet{Lind:etal:2011} approach allows us to consider MRF approximations
to the Matern-based GP for $\nu=1$ and $\nu=2$. Our second model
is the Lindgren approximation for Matern-based GPs with $\nu=1$.
To implement the Lindgren model, one modifies the $Q$ matrix defined
previously as follows. Let $a=4+\frac{1}{\rho^{2}}$. The diagonal
elements of $Q$ are $4+a^{2}$. The entries corresponding to cardinal
neighbors are $-2a$. Those for diagonal neighbors are $2$, and those
for 2nd-order cardinal neighbors are $1$. This extends the neighborhood
structure relative to the CAR model and parameterizes it as a function
of $\rho$.

The primary difference between the CAR and Lindgren models is that
the Lindgren model provides an additional degree of freedom by estimating
$\rho$. In particular $\rho$ allows us to estimate the locality
of the smoothing. As $\rho$ decreases, the model uses more and more
local data to estimate the compositional proportions at a given location,
effectively averaging the empirical proportions over smaller neighborhoods.
In general, the \cite{Lind:etal:2011} model will generally provide
for a smoother estimate than the CAR model \citep{Paci:2013}. 

To ensure that the $\sigma^{2}$ parameter is mathematically equivalent
between the two models, we reparameterize, producing our second model:
\[
\alpha_{p}\sim\mbox{N}\left(\mu_{p},\sigma_{p}^{2}\cdot\frac{4\pi}{\rho_{p}^{2}}Q(\rho_{p})^{-1}\right)
\]


We refer to this model as the \emph{SPDE model}.


\subsection{Prior Distributions}

\noindent The ICAR specification contains a set of hyperparameters
$\sigma_{p}^{2}$ for $p$ = $1,...,P$. Following \cite{Gelm:2006}
we use a uniform distribution on each $\sigma_{p}$ parameter, with
upper bound of {[}check max values{]}. For the SPDE model we also
have parameters $\mu_{p}$, which we give flat, non-informative priors
(truncated at $\pm10$), and $\rho_{p}$ which we give uniform priors
on the interval $(0.1,\exp(5))$. 


\subsection{Latent Variable Model\label{sub:Latent-Variable-Model}}

It is well-known that devising an effective MCMC algorithm for models
with latent Gaussian process(es) and a non-Gaussian likelihood is
difficult (cite recent cent/noncent papers \citep{Rue:Held:2005,Chri:etal:2006,Tan:Nott:2013}.
To develop an algorithm, we make use of a latent variable representation
for the multinomial probit model \citep{McCu:Ross:1994}. The representation
introduces latent variables that allow one to develop a MCMC sampling
strategy that takes advantage of closed form full conditional distributions
(so-called Gibbs sampling steps) for $\alpha_{p}$.

Suppose that compositional counts are available at a number of locations.
At location $i$, a sample size of $n_{i}$ observations are collected,
and each observation (i.e., each tree) can be classified into $P$
distinct categories. For a given tree $j$ at location $i$, let $Y_{ij}$
denote the response variable indicating the category. Let $Y_{ij}$
be associated with $P$ latent variables $W_{ij1},...,W_{ijP}$ such
that $Y_{ij}$ = $p$ if and only if $W_{ijp}={\displaystyle \max_{p'}\big\{ W_{ijp'}\big\}}$;
in other words, the maximum of the set of latent variables $\{W_{ijp}\}{\displaystyle _{p=1}^{P}}$
determines the category of observation $j$ at location $i$. 



\noindent The final piece of the latent variable representation is
the relationship between the $W$ variables and the $\alpha_{p}(s)$
processes. We have that

\noindent 
\[
W_{ijp}\sim\mbox{N}(\alpha_{p}(s_{i}),1)
\]
independently for all of the $W_{ijp}$ values. 

\noindent Consider the following example with two locations that are
neighbors and $P=2$ categories. Each tree $j$ at location $i$ is
associated with two variables $W_{ij1}$ and $W_{ij2}$, governed
by the latent variables $\alpha_{1i}$ and $\alpha_{2i}$, respectively.
Suppose that $\alpha_{1i}>\alpha_{2i}$ for a given location $i$.
Then this model implies that any tree $j$ is more likely to be labeled
1 than 2 at location $i$. The difference between $\alpha_{1i}$ and
$\alpha_{2i}$ explains the \textit{difference} in probability of
\textit{categories} 1 and 2 at location $i$, and the similarity between
$\alpha_{p1}$ and $\alpha_{p2}$ explains the \textit{correlation}
between the probabilities at \textit{locations} 1 and 2 for category
$p$.


\subsection{Model for township data\label{sub:Model-for-township}}

We developed an extension of the model described in previous sections
to account for data at a different aggregation than our core 8 km
grid. This extension introduces a new set of latent variables, one
per tree, that indicate the grid cell in which the tree is located.
These latent 'membership' variables, $c_{tj}$, for tree $j$ in township
$t$, $t=1,\ldots,T$ can be sampled within the MCMC as additional
unknown parameters. The prior for $c_{tj}$ is a discrete distribution
that puts mass proportional to the areal overlap between the township
in which the tree is located and the $m$ grid cells, with the priors
across different trees in a township being independent and with most
of the . 
\[
c_{tj}\sim\mbox{Multinom}(1,\{\psi_{1},\ldots,\psi_{m}\}).
\]
Because the townships overlap a limited number of grid cells, most
of the $\psi_{1},\ldots,\psi_{m}$ values are zero.

Using the latent variable representation, we have that $W_{tjp}\sim\mbox{N}(\alpha_{p}(s_{c_{tj}},1)$
for tree $j$ in township $t$. In updating the other parameters in
the model during the MCMC (specifically the $\alpha$ values), we
condition on the current values, $c_{tj}$, which provides a ``soft''
assignment of trees to grid cells that respects both the known township
in which the tree occurred and the uncertainty in which grid cell
the tree occurred.

Note that this prior has the unrealistic feature that it does not
represent our knowledge that the trees in a township would be distributed
more regularly across the area of the township than expected by such
an independence prior.


\subsection{Computation}

The \cite{McCu:Ross:1994} representation is convenient for MCMC sampling,
particularly in this high-dimensional spatial context, as it allows
us to draw from the posterior conditional distributions of the $W_{ijp}$
variables (these distributions are truncated normal) in closed form
and to draw the entire vector of latent process values for each taxon,
$\alpha_{p}$, as a single sample that respects the spatial dependence
structure for each taxon.

While the latent variable representation provides great advantages
in the MCMC sampling for each $\alpha_{p}$ compared to joint Metropolis
updates or updating each location individually, there is still strong
dependence between the hyperparameters, $\{\sigma_{p}^{2},\mu_{p},\rho_{p}\}$
and the latent process values (as well as between the latent process
values and the latent $W_{ijp}$ variables). To address the first,
we developed a 'cross-level' joint updating strategy in which we propose
$\phi\in\{\sigma_{p}\},p=1,\ldots,P$ (and for the SPDE model, $\phi\in\{\{\sigma_{p}\},\{\mu_{p}\},\{\rho_{p}\}\}),$
via a Metropolis-style random walk and then conditional on the proposed
value of $\phi^{*}$ propose $\alpha_{p}$ from its full conditional
distribution given $\phi^{*}$and the latent $W$ variables. This
is equivalent to sampling from the marginalized (with respect to $\alpha_{p}$)
distribution of $\phi$ conditional on $W_{p}$. Note that in sampling
$\sigma_{p}$and $\rho_{p}$ in the SPDE model, for each $p$, we
jointly propose $\{\sigma_{p},\rho_{p}\}$ using an adaptive Metropolis
proposal and then use the cross-level strategy to also propose $\alpha_{p}$
as part of a single accept-reject step. For these various joint samples
of hyperparameters and $\alpha_{p}$, we use adaptive Metropolis sampling
\citep{Shab:Well:2011}.

The full description of the MCMC sampling steps is provided in the
Appendix. In addition, in the latent variable representation, $\theta_{p}(s)$
never appears explicitly and cannot be calculated in close form. Instead
we use Monte Carlo integration over $W_{ijp},\, p=1,\ldots,P$ to
estimate $\theta_{p}(s_{i})$, also described in the Appendix. 

The model is implemented in R \cite{R:2014} with core computational
calculations coded in C++ using the \emph{Rcpp} package \citep{Edde:Fran:2011}.
We also make extensive use of sparse matrix representations and algorithms,
using the \emph{spam} package in R \citep{Furr:Sain:2010}. All code
is available on Github, including pre- and post-processing code, at
https://github.com/Paleon-Project/composition. 


\section{Model comparison}


\subsection{Design}

We compared the CAR and SPDE models using cross-validation by holding
out data from the fitting process and assessing the fit of the model
on the held-out data. This cross-validation used a subregion containing
most of Minnesota and a small amount of western Wisconsin, defined
to be the cells whose x-coordinate was less than 300,000 (this defines
a north-south line that approximately goes through Duluth, Minnesota)
and hereafter referred to as the ``Minnesota subregion'' . We used
two types of experiments:
\begin{enumerate}
\item We held out all the data from 95\% of the cells in the Minnesota subregion,
with cells selected at random. This was meant to assess the ability
of the model to interpolate from a sparse set of cells/townships and
mimics the limited data in Illinois and Indiana.
\item We held out 5\% of the trees from all of the trees in the dataset
(leaving aside the held-out Minnesota subregion cells). This was meant
to assess the ability of the model to estimate the composition in
cells in which data were available. 
\end{enumerate}
Finally, in a separate sensitivity analysis we instead left out 80\%
of the cells in Minnesota subregion at random. This variation on the
first test above was meant to indicate whether our model comparison
conclusions would be robust as the digitization process for Illinois
and Indiana progresses and provides us with increasingly dense data. 

There has been extensive work in the statistical literature on good
metrics to use to compare the predictive ability of models; these
metrics are referred to as scoring rules. In particular it is well-recognized
that predictive distributions should maximize sharpness subject to
calibration. That is the predictive distribution should be as narrow
as possible while being calibrated such that the observations are
consistent with the distribution \citep{Gnei:etal:2007}. When thinking
in terms of prediction intervals as summaries of the predictive distribution,
we seek intervals that are as narrow as possible while still covering
the truth the expected proportion of the time. 

Following the suggestions in \cite{Gnei:etal:2007}, we considered
the following metrics. For experiment 1, $Y_{i}=\{Y_{i1},\ldots,Y_{iP}\}$
is the count of all trees in the held-out cell and for experiment
2, $Y_{i}$ is the count of held-out individual trees in the cell.
For each of these metrics, we calculated the metric in two ways. First,
we used the posterior mean composition estimates (as a measure of
our core predictions), with $\tilde{\theta}_{p}(s)$ being the posterior
mean. Second, we averaged the metric over the posterior samples (as
a measure of our full data product, including uncertainty), taking
$\tilde{\theta}_{p}(s)$ to be an individual MCMC sample and then
averaging the metric over all the posterior samples. 
\begin{enumerate}
\item Brier score: \cite{Gnei:etal:2007} suggest this measure, which has
been in use for decades. For multi-category as opposed to binary outcomes,
this takes the form
\[
\frac{1}{n_{i}}\sum_{j=1}^{n_{i}}\sum_{p=1}^{P}(y_{ijp}-\tilde{\theta}_{p}(s_{i}))^{2}
\]
for each cell. The full Brier score for the complete set of held-out
locations sums over all locations.
\item Log predictive density: This measure takes the log of the density
of held-out observations under the fitted model, $Y_{i}\sim\mbox{Multinom}(n_{i},\{\tilde{\theta}_{1}(s_{i}),\ldots,\tilde{\theta}_{P}(s_{i})\})$,
summing on the log scale across all of the held-out data. 


While in principle, this metric should be optimal \citep{Krnj:Drap:2014},
it suffers from a lack of robustness in being very sensitive to small
predictions near zero \citep{Gnei:etal:2007}. Even worse, our Monte
Carlo estimation of $\theta$ used 10000 samples, so in some cases
$\tilde{\theta}_{p}(s)=0$. When a tree is present in a cell but its
corresponding proportion is 0, this gives a log density of $-\infty$,
preventing use of the metric. As an informal solution to this we set
$\tilde{\theta}_{p}(s)=\frac{1}{100000}$ in such cases, but given
these issues we treat the log predictive density as a secondary measure.

\item (Experiment 1 only) Weighted root mean square prediction error (RMSPE)
and mean absolute error (MAE), averaging over cells and taxa. We weight
by the number of held-out trees in each cell to account for the greater
variability in the empirical proportions in locations with few held-out
trees. 
\item (Experiment 1 only) Coverage and length of 95\% prediction intervals
for $Y_{ip}$. We considered only cells with at least 50 trees to
focus our assessment on cases where the raw estimates, $\hat{\theta}_{p}(s)$,
were reasonably certain and avoid being strongly influenced by predictive
inference for cells where observational variability dominates.
\end{enumerate}
Note that for the calculation of the metric applied to the individual
posterior samples, we can estimate the posterior probability that
one model has a lower (better) value of the metric than the other
model by simply calculating the proportion of samples for which each
model has a lower value of the metric.

In our initial exploratory fitting, we noticed that the SPDE model
produced boundary effects in the predicted composition near the edges
of the convex hull of the observations. To attempt to alleviate this,
we added a buffer zone of six grid cells around our entire original
domain, although the boundary effects were still evident even after
inclusion of the buffer. For the model comparison, for comparability,
we included this buffer for both the SPDE and CAR models. 

For the model comparison, we used an earlier version of the dataset
{[}include text if we stick with 0.5 for CV{]}. Given that part of
the goal of the exercise is to understand the ability of the models
to interpolate to areas without data and the ongoing nature of digitization,
we believe this difference in datasets is not an issue.

We ran each model for 150,000 iterations. After discarding 25,000
iterations for burn-in, we retained a posterior sample of 250 subsampled
iterations to reduce post-processing computations and storage needs. 





\subsection{Results}

Here we summarize the results of our cross-validation analyses that
inform the choice between the CAR and SPDE models. 


\subsubsection{Full cell hold-out experiment}

Table \ref{tab:score_cell_fivepercent} shows predictive ability for
Experiment 1: those cells in the Minnesota subregion held out of the
fitting process, for both the mean of the metric applied to the individual
posterior samples and for the metric applied to the posterior mean
predictions. Table \ref{tab:coverage_cell_fivepercent} shows performance
of the uncertainty estimates. For the posterior distribution over
the predictive metric values, the CAR model outperforms the SPDE model,
while for the posterior mean predictions, the SPDE model appears to
outperform the CAR model, but we do not have any uncertainty estimates
for this comparison. Coverage and interval lengths are similar between
the two models. From a practical perspective, based on the difference
in mean absolute error, which is on a readily interpretable scale,
the differences between the models are small. 




\begin{table}
\caption{Predictive ability based on several predictive metric criteria for
the CAR and SPDE spatial models when holding out 95\% of entire cells
of data in Minnesota. Smaller values are better.}


\begin{tabular}{|c|c|c|>{\centering}p{3cm}|>{\centering}p{2.5cm}|>{\centering}p{2.5cm}|}
\hline 
 &
\multicolumn{3}{c|}{{\small{Posterior mean of metric}}} &
\multicolumn{2}{c|}{{\small{Metric of posterior mean predictions}}}\tabularnewline
\hline 
\hline 
 &
{\small{CAR model}} &
{\small{SPDE model}} &
{\small{Posterior Prob. CAR < SPDE}} &
{\small{CAR model}} &
{\small{SPDE model}}\tabularnewline
\hline 
{\small{Brier}} &
{\small{180688}} &
{\small{186045}} &
{\small{0.98}} &
{\small{162813}} &
{\small{161559}}\tabularnewline
\hline 
{\small{Negative Log Density}} &
{\small{468397}} &
{\small{513144}} &
{\small{1.00}} &
{\small{395685}} &
{\small{395998}}\tabularnewline
\hline 
{\small{Mean Absolute Error}} &
{\small{0.0364}} &
{\small{0.0383}} &
{\small{0.98}} &
{\small{0.0275}} &
{\small{0.0269}}\tabularnewline
\hline 
{\small{Root Mean Square Error}} &
{\small{0.0897}} &
{\small{0.0959}} &
{\small{0.96}} &
{\small{0.0647}} &
{\small{0.0626}}\tabularnewline
\hline 
\end{tabular}

\label{tab:score_cell_fivepercent}
\end{table}


\begin{table}
\caption{Coverage and length of prediction intervals for the CAR and SPDE spatial
models when holding out 95\% of entire cells of data in Minnesota. }


\begin{tabular}{|c|c|c|}
\hline 
 &
{\small{CAR model}} &
{\small{SPDE model}}\tabularnewline
\hline 
{\small{Coverage}} &
{\small{0.976}} &
{\small{0.978}}\tabularnewline
\hline 
{\small{Mean Interval Length}} &
{\small{0.129}} &
{\small{0.142}}\tabularnewline
\hline 
{\small{Median Interval Length}} &
{\small{0.037}} &
{\small{0.033}}\tabularnewline
\hline 
\end{tabular}

\label{tab:coverage_cell_fivepercent}
\end{table}




Table \ref{tab:score_cell_20percent} shows the results when the proportion
of cells that are held out decreases from 95\% to 80\%, indicating
performance on less sparse data. Table \ref{tab:coverage_cell_20percent}
shows performance of the uncertainty estimates. For denser data, unlike
the results when data are more sparse, the SPDE model generally outperforms
the CAR model, but again differences from a practical perspective,
based on mean absolute error, are limited.



\begin{table}
\caption{Predictive ability based on several predictive score criteria for
the CAR and SPDE spatial models when holding out 80\% of entire cells
of data in Minnesota. Smaller values are better, except for coverage,
where values near 0.95 are optimal.}


\begin{tabular}{|c|c|c|>{\centering}p{3cm}|c|c|}
\hline 
 &
\multicolumn{3}{c|}{{\small{Posterior mean of score}}} &
\multicolumn{2}{c|}{{\small{Score of posterior mean predictions}}}\tabularnewline
\hline 
\hline 
 &
{\small{CAR model}} &
{\small{SPDE model}} &
{\small{Posterior Prob. CAR < SPDE}} &
{\small{CAR model}} &
{\small{SPDE model}}\tabularnewline
\hline 
{\small{Brier}} &
{\small{141546}} &
{\small{140211}} &
{\small{0.10}} &
{\small{130006}} &
{\small{129910}}\tabularnewline
\hline 
{\small{Negative Log Density}} &
{\small{356985}} &
{\small{355640}} &
{\small{0.31}} &
{\small{312509}} &
{\small{312950}}\tabularnewline
\hline 
{\small{Mean Absolute Error}} &
{\small{0.0309}} &
{\small{0.0297}} &
{\small{0.10}} &
{\small{0.0225}} &
{\small{0.0222}}\tabularnewline
\hline 
{\small{Root Mean Square Error}} &
{\small{0.0763}} &
{\small{0.0740}} &
{\small{0.04}} &
{\small{0.0531}} &
{\small{0.0528}}\tabularnewline
\hline 
\end{tabular}

\label{tab:score_cell_20percent}
\end{table}


\begin{table}
\caption{Coverage and length of prediction intervals for the CAR and SPDE spatial
models when holding out 80\% of entire cells of data in Minnesota. }


\begin{tabular}{|c|c|c|}
\hline 
 &
{\small{CAR model}} &
{\small{SPDE model}}\tabularnewline
\hline 
{\small{Coverage}} &
{\small{0.981}} &
{\small{0.971}}\tabularnewline
\hline 
{\small{Mean Interval Length}} &
{\small{0.112}} &
{\small{0.103}}\tabularnewline
\hline 
{\small{Median Interval Length}} &
{\small{0.028}} &
{\small{0.021}}\tabularnewline
\hline 
\end{tabular}

\label{tab:coverage_cell_20percent}
\end{table}



\subsubsection{Individual tree hold-out experiment}

Table \ref{tab:score_tree} shows the predictive ability for Experiment
2. Here we have evidence (posterior probability of 0.94) that the
SPDE model is better based on the Brier score.



\begin{table}
\caption{Predictive ability based on several predictive score criteria for
the CAR and SPDE spatial models when holding out 5\% of trees. Smaller
values are better.}


\begin{tabular}{|c|c|c|>{\centering}p{3cm}|c|c|}
\hline 
 &
\multicolumn{3}{c|}{{\small{Posterior mean of metric}}} &
\multicolumn{2}{c|}{{\small{Metric of posterior mean predictions}}}\tabularnewline
\hline 
\hline 
 &
{\small{CAR model}} &
{\small{SPDE model}} &
{\small{Posterior Prob. CAR < SPDE}} &
{\small{CAR model}} &
{\small{SPDE model}}\tabularnewline
\hline 
{\small{Brier}} &
{\small{21296}} &
{\small{21277}} &
{\small{0.06}} &
{\small{21146}} &
{\small{21143}}\tabularnewline
\hline 
{\small{Negative Log Density}} &
{\small{52199}} &
{\small{52052}} &
{\small{0.01}} &
{\small{51141}} &
{\small{51156}}\tabularnewline
\hline 
\end{tabular}

\label{tab:score_tree}
\end{table}



\subsubsection{Choice of spatial model}

The differences between models are not consistent across the various
comparisons, so there is not a clear choice. In our final data product
we use the CAR model, for three reasons. First, the CAR model has
modestly better performance when data are sparse, as is still the
case for Illinois and Indiana. Second, the model is simpler and easier
to explain and computations can be done more quickly. Third, predictions
from the SPDE model showed boundary effects, with some taxa showing
non-negligible posterior mean values at the edges of the domain, well
away from where the taxa were present in the empirical data. This
included non-negligible values within (but near the edge of) the convex
hull of locations with data. 


\section{Data product}

For the final data product, we ran the model using the CAR specification
for 150,000 iterations, discarding the first 25,000 iterations for
burn-in and retaining 250 subsampled iterations to limit storage needs
and limit the size of the data product. Mixing was generally reasonable,
but for some of the hyperparameters was relatively slow, particularly
for less common taxa. Despite this, mixing for the variables of substantive
interest, the proportions was good. {[}Report ESS values{]}

In Figure \ref{fig:select_maps}, we show maps of estimated composition
for the full domain for several taxa of substantive interest to illustrate
the results: beech, cherry, chestnut, elm, hemlock, oak, and pine.
These maps contrast the raw data proportions, the posterior means
and posterior standard deviations as pointwise estimates of uncertainty.
Fig. \ref{fig:all_predictions} shows posterior means for all 23 taxa.

\begin{figure}
\label{fig:select_maps}

\includegraphics[scale=0.75]{1_accounts_gen_vis_paciorek_research_jmac_composition_paper_fig2.pdf}

\hspace{4mm}\includegraphics[scale=0.16]{2_accounts_gen_vis_paciorek_research_jmac_composition_paper_legendRaw.pdf}\hspace{4mm}\includegraphics[scale=0.16]{3_accounts_gen_vis_paciorek_research_jmac_composition_paper_legendProp.pdf}\hspace{3.5mm}\includegraphics[scale=0.16]{4_accounts_gen_vis_paciorek_research_jmac_composition_paper_legendSD.pdf}

\caption{Empirical proportions from raw data (column 1), predictions in the
form of posterior means (column 2) and uncertainty estimates in the
form of posterior standard deviations -- representing standard errors
of prediction (column 3) for select taxa.}
\end{figure}


\begin{figure}
\label{fig:all_predictions}

\includegraphics[scale=0.9]{5_accounts_gen_vis_paciorek_research_jmac_composition_paper_fig3.pdf}

\caption{Predictions (posterior means) for all taxa over the entire domain.}


\end{figure}


The data product is publicly available at XX (insert accession number)
under YY license as version 0.3 as of April 2015. The product is in
the form of a netCDF-4 file, with dimensions x, y, and MCMC iteration.
There is one variable per taxon. The PalEON project will continue
to maintain this product, releasing new versions as additional data
in Illinois, Indiana and Ohio are digitized. 

{[}comment on stability of hosting once get better feel for this{]}


\section{Discussion}

Note that digitization of data from Illinois and Indiana is ongoing,
and digitization of additional data from Ohio is planned as well.
As a result, at some point we expect to have complete data for the
western half of the domain. There will remain some missing townships
in the eastern half of the domain. 

Given the density of data and the limited differences seen between
the CAR and SPDE models, we expect the data product to be reasonably
robust to the choice of spatial model, particularly in those areas
with complete data. However, additional investigation of other statistical
representations is of interest, in particular nonstationary spatial
models and use of covariates. The biggest shortcoming of the current
model is its inability to account for local features such as rivers
(note the Minnesota River floodplain in evidence in the raw data).
The current model, by using a simple stationary spatial model that
smooths as a function of Euclidean distance, does not account for
topographic, soil, or other features. 

Other related data products that are under development include:
\begin{itemize}
\item The gridded raw count data for the western subdomain and data aggregated
to township for the eastern subdomain. The raw data product is available
at XXX and that product provides the input data for the product described
in this work. Note that we cannot provide the raw data at point locations
because of limitations specified in the data use agreements under
which we have access to the data.
\item Gridded raw biomass estimates for Minnesota, Wisconsin, and Michigan
based on the PLS data, with extension to Illinois and Indiana planned.
\item Statistically estimated biomass for Minnesota, Wisconsin, and Michigan
using a statistical model applied to the raw biomass estimates, with
extension to Illinois and Indiana planned.
\end{itemize}
An additional drawback of the product is its focus on composition,
which does not directly tell us about vegetation structure, in particular
does not distinguish between closed forest, savanna, and prairie,
of particular note in Minnesota, Wisconsin, Illinois, and into Indiana.
The statistically-estimated biomass product mentioned just above will
directly inform questions about vegetation structure. Extensions of
that product will also estimate basal area and stem density. 


\section*{Acknowledgments}

The authors are deeply indebted to all of the researchers over the
years who have preserved, collected, and digitized survey records,
in particular Jim Dyer, Peter Marks, Robert McIntosh, and Ed Schools
\{David/Simon please add names here\}\{Jason/Jody, should we mention
Bowles and Stuckey; if so I need full names\}. We thank Madeline Ruid,
Ben Seliger, Morgan Ross and Daniel Handel for processing of the southern
Michigan data. Indiana and Illinois data were made possible through
the hard work of many Notre Dame undergraduates in the McLachlan lab.
This work was carried out by the PalEON Project with support from
the National Science Foundation MacroSystems Program through grants
EF-1065656, EF-1241868, DEB-1241874 and DEB-1241868 and from the Notre
Dame Environmental Change Initiative. 

%\bibliographystyle{/accounts/gen/vis/paciorek/latex/RSSstylefile/Chicago}
\bibliographystyle{/accounts/gen/vis/paciorek/latex/PLOS_ONE/plos215}

\bibliography{/accounts/gen/vis/paciorek/bibfiles/abbrev.stat,/accounts/gen/vis/paciorek/bibfiles/abbrev.other,/accounts/gen/vis/paciorek/bibfiles/spatstat,/accounts/gen/vis/paciorek/bibfiles/statgeneral,/accounts/gen/vis/paciorek/bibfiles/paciorek,/accounts/gen/vis/paciorek/bibfiles/ecology,/accounts/gen/vis/paciorek/research/jmac/composition_paper/goring_intro}


\section{Appendix}


\subsection{MCMC details}

Define $\bar{w}_{i\cdot p}=n_{i}^{-1}{\displaystyle \sum_{j=1}^{n_{i}}W_{ijp}}$
be the average of the $W$ values for the $p$th taxon in the $i$th
grid cell. Let $A$ be a diagonal matrix where $A_{ii}$ is the number
of trees in the $i$th grid cell. When there are no trees in a grid
cell, $\bar{w}_{i\cdot p}=0$ and $A_{ii}=0$. For the township data,
at each iteration, based on the current values of the grid cell membership
variables, $c_{tj}$, trees are aggregated into grid cells and the
calculations above can then be carried out.

The conditional distribution for $W_{ijp}$ given the other unknowns
in the model and the data is as follows. Let $\mbox{TN}(a,b,\mu,\tau^{2})$
denote the truncated normal distribution with mean parameter $\mu$
and variance parameter $\tau^{2}$, truncated below by $a$ and above
by $b$. 

\begin{align}
W_{ijp} & \sim\begin{cases}
\mbox{TN}\big({\displaystyle \max_{p^{*}\neq y_{ij}}w_{ijp^{*}},\infty,\alpha_{y_{ij}}(s_{i}),1\big),} & \mbox{if }p\mbox{ = \ensuremath{y_{ij}}}\\
\mbox{TN}\big(-\infty,w_{ijy_{ij}},\alpha_{p}(s_{i}),1\big), & \mbox{if }p\mbox{ \ensuremath{\neq y_{ij}}}
\end{cases}\label{sampW}
\end{align}
In essence, the truncation value is determined by the taxon of the
$j$th tree. For a given $p$, the $W$ values for all trees in all
cells can be sampled in parallel. 

The conditional distribution of $\alpha_{p}$ is 
\begin{align}
\alpha_{p} & \sim\mbox{N}\bigg(\Big(A+Q_{p}\Big)^{-1}A\bar{w}_{p},\Big(A+Q_{p}\Big)^{-1}\bigg).\label{sampalpha}
\end{align}
where $Q_{p}=(\sigma_{p}^{2})^{-1}Q$ for the CAR model and $\left(\sigma_{p}^{2}\cdot\frac{4\pi}{\rho_{p}^{2}}\right)^{-1}Q(\rho_{p})$
for the SPDE model. For each hyperparameter, $\phi\in\{\{\mu_{p}\},\{\log\sigma_{p},\rho_{p}\}\}$,
we sampled $\{\phi,\alpha_{p}\}$ jointly, proposing $\phi$ as a
random walk and, conditional on the proposed value of $\phi$, sampling
$\alpha_{p}$ from the distribution just above. The joint proposal
is accepted or rejected as a standard Metropolis-Hastings proposal,
with adaptation of the proposal (co)variance \citep{Shab:Well:2011}.
As mentioned previously, for the SPDE model, we jointly proposed $\phi=\{\log\sigma_{p},\rho_{p}\}$
from a bivariate normal distribution to account for the posterior
dependence of these parameters, while for the CAR model, $\phi\in\{\{\log\sigma_{p}\}\}$. 

\noindent 

For the township-level data, for a given tree $j$ in township $t$,
we draw the latent tree membership variable, $c_{tj}\in\{1,\ldots,m\}$,
from a discrete distribution by normalizing posterior weights, $\{\psi_{1}L_{1},\ldots,\psi_{m}L_{m}\}$,
produced by multiplying the prior weights by a likelihood contribution.
For cell $k$, $L_{k}$ is the density of the latent $W_{tj1},\ldots,W_{tjP}$
values for the given tree under the condition that $c_{tj}=k$, namely
the product of independent normal densities, $W_{tjp}\sim\mbox{N}(\alpha_{p}(s_{k}),1)$,
over $p=1,\ldots,P$. Thus the posterior reweights the prior based
on how consistent the current $W$ values for a tree are with the
$\alpha$ values for the candidate grid cells. 




\subsection{Estimating $\theta_{p}(s)$ via Monte Carlo integration}

In the latent variable representation, $\theta_{p}(s)$ never appears
explicitly and cannot be calculated in close form. Instead we use
Monte Carlo integration over $W_{ijp},\, p=1,\ldots,P$ to estimate
$\theta_{p}(s_{i})$. The quantity $\theta_{p}(s_{i})=P(W_{ijp}={\displaystyle \max_{p^{*}}W_{ijp^{*}})}$
defines the probability of taxon $p$ at grid cell $i$. This requires
one to choose the number of Monte Carlo samples, which we set at 10000.
For each of the saved MCMC samples, $k=1,\ldots K$, we estimate $\theta_{p}^{(k)}(s_{i})$
numerically. Specifically, for $t=1,\ldots,10000$, we independently
draw
\[
W_{itp}^{(k)}\sim\mbox{N}(\alpha_{p}^{(k)}(s_{i}),1),\, p=1,\ldots,P
\]
and estimate 
\[
\theta_{p}^{(k)}(s_{i})\approx\frac{1}{10000}{\displaystyle \sum_{t=1}^{10000}1(W_{itp}^{(k)}={\displaystyle \max_{p^{*}}W_{itp^{*}}^{(k)})}}
\]
where $1(\cdot)$ is the indicator function that evaluates to 1 if
the expression is true and 0 if false. In other words, we calculate
the proportion of times that the maximum of $W_{itp},\, p=1,\ldots,P$
corresponds to taxon $p$. Considering $\theta_{p}^{(k)}(s_{i}),\, k=1,\ldots,K$,
we have a sample from the posterior of $\theta_{p}(s_{i})$.

\end{document}
